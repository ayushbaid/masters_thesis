
\newcommand{\etas}{\ensuremath{\eta_{\mathrm{s}}}}


\chapter{Introduction}

Laparoscopy is a popular \textit{minimally invasive surgery} technique in which operations are performed by inserting equipments through small incisions. Laparoscopy surgery has advantage such as less pain and hemorrhaging, shorter recovery times over open procedures. The key equipment is a \textbf{laparoscope}, an optical imaging instrument which relays the visuals on a screen. Another main equipment is a cold light source to illuminate the area of operation.

The closed nature of laparoscopy images presents some challenges. The images can get severely corrupted with specular highlights \cite{stoyanov2005removing, saint2011detection}, surgical smoke \cite{barrett2003surgical}, and noise. Specular highlights result from strong reflection of the light source by body fluids like blood and mucus. Speckles interfere with post-processing like segmentation \cite{prokopetc2015segmentation, voros2007segmentation} and tracking \cite{wolf2011tracking}. Electrical cauterization of a tissue generates surgical smoke, which hinders visibility for surgeons and robots alike. Noise is present in all optical imaging systems and a laparoscope is no exception.

Chapter 2 covers the related work in this area. Recent work in specular highlight removal in laparoscopy use anisotropic diffusion and does not preserve texture and edges. There is lack of literature for smoke removal in laparoscopy images, and hence we look into haze removal in outdoor images. These methods lack some propereties of laparoscopy images which can be imposed due to the smaller variation in laparoscopy images than outdoor images.

In Chapter 3, we will introduce the image formation and model the variables in the system using priors on probabilistic graphical models. We formulate a unified Bayesian inference problem in Chapter 4, which is solved using expectation-maximation (EM) algorithm. We introduce variational Bayesian approximation to overcome the analytical intractability in the optimization scheme. We will present two versions of our algorithm, VBEM1 and VBEM2, where the second version is to make our method more sound in mathematical sense.

We present the validation metric and results of our algorithm compared to the state-of-the-art in Chapter 5. We conclude and discuss some future directions in Chapter 6.





%%


%%% Local Variables: 
%%% mode: latex
%%% TeX-master: "../mainrep"
%%% End: 
