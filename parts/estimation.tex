\chapter{Estimation}

For our algorithm, we assume that we are provided with a speckle label map $\mathbf{r}$, speckle color $K_\text{spec}$, and smoke color $K_\text{smoke}$. We will use maximum aposteriori probability (MAP) estimation scheme for image $\mathbf{X}$. Given the speckle map $\mathbf{r}$, we remove the pixel locations of $\mathbf{y}$ where which are specular highlights, as they convey no information about the underlying image. We call the new image with holes in place of specular highlights as $\mathbf{\mathring y}$. The MAP estimate is

\begin{align}
\hat{\mathbf{x}} &= \argmax_\mathbf{x} P \left( \mathbf{x} | \mathbf{\mathring y}, \mathbf{r} \right) = \argmax_\mathbf{x} P \left( \mathbf{\mathring y} | \mathbf{x}, \mathbf{r} \right) P \left( \mathbf{x} \right) \label{eqn:simplemap}
\end{align}

The first part of the final term in \Eqref{eqn:simplemap} is the likelihood of observing the output $\mathbf{\mathring y}$. The second part is the prior probability of the image estimate $\mathbf{x}$.


\section{Variational Bayesian expectation maximization}
We will introduce dictionary coding $\mathbf{S}$ and smoke transmission map $\mathbf{T}$ as latent variables in the system. The posterior probability is 
\begin{align}
P \left( \mathbf{\mathring y} | \mathbf{x}, \mathbf{r} \right) P \left( \mathbf{x} \right) &= \int_{\mathbf{s}, \mathbf{t}} P \left( \mathbf{\mathring y}, \mathbf{x}, \mathbf{s}, \mathbf{t} | \mathbf{r} \right) \diff \mathbf{s} \diff\mathbf{t} \\
&= \int_{\mathbf{s}, \mathbf{t}} P \left( \mathbf{\mathring y} | \mathbf{x}, \mathbf{s},  \mathbf{t}, \mathbf{r} \right)  P \left(\mathbf{x}, \mathbf{s} \right) P \left( \mathbf{t} \right) \diff\mathbf{s} \diff\mathbf{t} \label{eqn:mapDerive1}\\
&= \int_{\mathbf{s}, \mathbf{t}} P \left( \mathbf{\mathring y} | \mathbf{x}, \mathbf{t}, \mathbf{r} \right)  P \left(\mathbf{x}, \mathbf{s} \right) P \left( \mathbf{t} \right) \diff\mathbf{s} \diff\mathbf{t}  \label{eqn:mapDerive2}
\end{align}
\Eqref{eqn:mapDerive2} follows as $\mathbf{Y}$ is independent of $\mathbf{S}$, given $\mathbf{T}$ and $\mathbf{X}$ due to our image formation model.
The final term in \Eqref{eqn:mapDerive2} has three components. The first one is the likelihood of the output being observed. The second and third component are priors on $\mathbf{X}, \mathbf{S}$ and $\mathbf{T}$.

Due to our modeling of the noise as i.i.d. Gaussian, the log likelihood probability distribution turns out to be
\begin{align}
\log P \left( \mathbf{\mathring y}| \mathbf{x}, \mathbf{t}, \mathbf{r} \right) &= - \frac{1}{\sigma^2} \sum_{i | r_i = 0} \left( y_i - t_i x_i - (1-t_i) K_\text{smoke} \right)^2
\end{align}

We will now cover the expectation and maximization steps in the EM algorithm. Variational Bayesian factorization is introduced in the expectation step to overcome intractability.


\subsection{Expectation step}
Let $\mathbf{x}^m$ be the estimate of the uncorrupted image after $m$ iterations. The $Q$ function defined for the next iteration is
\begin{align}
Q(\mathbf{x}; \mathbf{x}^m) &= \mathbb{E}_{ P \left( \mathbf{S}, \mathbf{T} | \mathbf{\mathring y}, \mathbf{x}^m, \mathbf{r} \right)} \left[ \log P \left( \mathbf{\mathring y}, \mathbf{x}, \mathbf{S}, \mathbf{T} | \mathbf{r} \right)\right] \nonumber \\
&= \mathbb{E}_{ P \left( \mathbf{S}, \mathbf{T} | \mathbf{\mathring y}, \mathbf{x}^m, \mathbf{r} \right)} \left[ P \left( \mathbf{\mathring y}, \mathbf{T}| \mathbf{x}, \mathbf{r} \right) P \left( \mathbf{S} | \mathbf{x} \right) P \left( \mathbf{x} \right) \right] \label{eqn:Qvbem2}
\end{align}

The posterior probability over the latent variables can be written as
\begin{align}
P \left( \mathbf{S}, \mathbf{T} | \mathbf{y}, \mathbf{x}^m, \mathbf{r} \right) &= P \left( \mathbf{T} | \mathbf{y}, \mathbf{x}^m, \mathbf{r} \right) P \left( \mathbf{S} | \mathbf{x}^m \right) \label{eqn:latentPosterior}
\end{align}

The codes $S_i$ for $I'$ full patches in $\mathbf{X}$ are independent of each other. Hence
\begin{align}
P \left( \mathbf{S} | \mathbf{x}^m\right) &= \prod_{i=1}^{I'} P \left( S_i | \mathbf{x}^m \right) \label{eqn:obCodeFact} 
\end{align}

The simplified expression for $Q\left( \cdot \right)$ is
\begin{align}
Q(\mathbf{x}; \mathbf{x}^m) = &\mathbb{E}_{ P \left( \mathbf{T} | \mathbf{\mathring y}, \mathbf{x}^m, \mathbf{r} \right)} \log \left[ P \left( \mathbf{\mathring y}, \mathbf{T} | \mathbf{x}, \mathbf{r} \right) \right] 
+ \sum_{i=1}^{I'} \mathbb{E}_{ P \left( S_i | \mathbf{x}^m \right)} \left[ \log P \left( S_i | \mathbf{x} \right) \right]
+ \log P \left( \mathbf{x} \right) \label{eqn:QSimplified} 
\end{align}

The expectations in \Eqref{eqn:QSimplified} are analytically intractable. We use variational factorization for the latent variables $\mathbf{S}$ and $\mathbf{T}$. The posterior probability for $\mathbf{T}$ is factorized over each pixel in the image, and the posterior probability for $\mathbf{S}$ is factorized over coefficient corresponding to each of the $J$ atoms of dictionary $\mathbf{D}$.
\begin{align}
P \left( \mathbf{T} | \mathbf{\mathring y}, \mathbf{x}^m, \mathbf{r} \right) &\approx \prod_{i=1}^{I} G^\dagger_i \left( T_i | \mathbf{\mathring y}, \mathbf{x}^m, \mathbf{r} \right) \label{eqn:vbFactT} \\
P \left( S_i | \mathbf{x}^m \right) &\approx \prod_{j}^{J} G_{ij} (S_{ij} | \mathbf{x}^m) \label{eqn:vbFactS}
\end{align}

$G^\dagger_i$ is the factor of $\mathbf{T}$ at pixel $i$, and $G_{ij}$ is the factor corresponding to $j^{th}$ coefficient of code vector $S_i$. $\lbrace G^\dagger_i \rbrace_{i=1}^{I}$ are truncated Gaussians with support $\left[0, 1 \right]$, means $\lbrace \mu^\mathbf{T}_i \rbrace$,  standard deviations $\lbrace \sigma^\mathbf{T}_i \rbrace$. $\lbrace G_{ij} \rbrace_{j=1}^{J}$ are Gaussian distributions with means $ \lbrace \mu^{\mathbf{S}}_{ij} \rbrace $ and standard deviation $ \lbrace \sigma^{\mathbf{S}}_{ij} \rbrace $. We have to optimize the factorization before solving for the $Q$ function. Given all other factors, the optimum factor for $G^{\dagger*}_i$ and $G^*_{ij}$ are derived as
\begin{align}
\log G^{\dagger*}_i (T_i) &= \mathbb{E}_{ \prod_{k \neq i} G^{\dagger*}_k \left( T_k | \mathbf{\mathring y}, \mathbf{x}^m, \mathbf{r} \right) } \left[ \log P \left( \mathbf{\mathring y}, \mathbf{T} | \mathbf{x}^m, \mathbf{r} \right) \right] + c_1 \label{eqn:solveFactT} \\
\log G_{ij}^* \left( S_{ij} \right) &= \mathbb{E}_{ \prod_{k \neq j} G^*_{ik} \left( S_{ik} | \mathbf{x}^m \right) } \left[ \log P \left( S_i | \mathbf{x}^m \right)\right] + c_2 \label{eqn:solveFactS}
\end{align}

\Eqref{eqn:solveFactT} and \Eqref{eqn:solveFactS} are solved iteratively for optimum factors $ \lbrace G^{\dagger*}_i \rbrace $ and $ \lbrace G_{ij}^* \rbrace $ until convergence. $c_1$ and $c_2$ are additive constants which will be absorbed in normalization of factor distributions. The parameters of optimum factors are derived in \Appref{app:factorParamsT} and \Appref{app:factorParamsS}.  Once we have the optimum factorization, solving for the $Q$ function becomes simple. We substitue $\mu^\mathbf{T}_i$ for $T_i$, $\left( \mu^\mathbf{T}_i \right)^2 + \left( \sigma^\mathbf{T}_i \right)^2$ for $T_i^2$ and $\mu^\mathbf{T}_i \mu^\mathbf{T}_j $ for $T_i T_j$, and $\mu^\mathbf{S}_{ij}$ for $S_{ij}$ in
\begin{align}
Q(\mathbf{x}; \mathbf{x}^m) = \mathbb{E}_{ \prod_{i=1}^{I} F^*_i (T_i | \mathbf{y}, \mathbf{x}^m) } \left[ \log P \left( \mathbf{y}, \mathbf{T} | \mathbf{x}, \mathbf{r} \right) \right] + \sum_i \mathbb{E}_{ \prod_{j=1}^{J} G^*_{ij} \left( S_{ij} | \mathbf{x}^m \right) } \left[ \log P \left( S_i | \mathbf{x} \right) \right] + \log P \left( \mathbf{x} \right)
\end{align}


\subsection{M step}
The final expression for $Q \left( \cdot \right)$ is
\begin{equation}
\begin{split}
Q\left( \mathbf{x}; \mathbf{x}^m \right) =  &-0.5 \gamma_1 \mathbf{\bar{x}}^\intercal_i \left( \mathbf{\bar{x}}_i -  \mathbf{D} \mu^\mathbf{S}_i \right) - \log \left( \gamma_2 J_\text{dist}\left( \mathbf{x} \right) \right) \\
&- \frac{1}{2 \sigma^2} \sum_{\{i | r_i = 0 \}}  \left(
\left(\mu^\mathbf{T}_i\right)^2 + \left(\sigma^\mathbf{T}_i\right)^2\right) \left(x_i - K_\text{smoke}\right)^2 \\
&- 2 \left(y_i - K_\text{smoke}\right) \mu^\mathbf{T}_i \left(x_i - K_\text{smoke} \right) + \delta'
\end{split}
\end{equation}
where $\delta'$ is independent of $\mathbf{x}$.

In this step, we will maximize $Q(\mathbf{x}; \mathbf{x}^m)$ with respect to $\mathbf{x}$ to obtain a new estimate $\mathbf{x}^{m+1}$. The next estimate is given by
\begin{align}
    \mathbf{x}^{m+1} &= \argmax_{\mathbf{x}} Q \left( \mathbf{x}; \mathbf{x}^{m+1} \right) \label{eqn:mStep}
\end{align}

This optimization problem is solved using projected gradient descent with adaptive step size.



